\documentclass[xcolor={dvipsnames}]{beamer}
\usepackage[utf8]{inputenc}
\usetheme{Rochester}
\beamertemplatenavigationsymbolsempty
\setbeamercolor*{structure}{bg=Mahogany!20,fg=Mahogany}

\setbeamercolor*{palette primary}{use=structure,fg=white,bg=structure.fg}
\setbeamercolor*{palette secondary}{use=structure,fg=white,bg=structure.fg!75}
\setbeamercolor*{palette tertiary}{use=structure,fg=white,bg=structure.fg!50!black}
\setbeamercolor*{palette quaternary}{fg=white,bg=black}

\setbeamercolor{section in toc}{fg=black,bg=white}
\setbeamercolor{alerted text}{use=structure,fg=structure.fg!50!black!80!black}

\setbeamercolor{titlelike}{parent=palette primary,fg=structure.fg!50!black}
\setbeamercolor{frametitle}{bg=Mahogany,fg=white}

\setbeamercolor*{titlelike}{parent=section in toc}
% \usecolortheme{beaver}

%% Paquetes adicionales
\usepackage{multicol}
\usepackage{ragged2e}
\usepackage{tikz}
\usepackage{minted}
\usepackage{setspace}

\title[Docker for dummies] {Docker for dummies}
\author[Leandro Kollenberger]
{Leandro Kollenberger}
\institute[redbee] {
redbee studios
}
\date[dockerpres] {
	30 de Marzo de 2018
}
\subject {Docker for dummies}
\titlegraphic{\vspace{-2cm}}
% \titlegraphic{\includegraphics[height=.5\textheight]{assets/docker_logo.png}}

\pgfdeclareimage[height=0.5cm]{left-logo}{assets/redbee_logo.png}

\setbeamertemplate{sidebar left}
{
\logo{\pgfuseimage{left-logo}}
\vfill%
\rlap{\hskip0.1cm\insertlogo}%
%\vskip15pt%
\vskip5pt%
}

\begin{document}

\begin{frame}
	 \tikz [remember picture,overlay]
		\node at
				% ([yshift=3cm]current page.south) 
				([yshift=1.8cm]current page.center)
				{\includegraphics[height=.5\textheight]{assets/docker_logo.png}};
	 \titlepage
\end{frame}

% \frame{\titlepage}

\begin{frame}[fragile]
	\frametitle{¿Qué es Docker?}
	\vspace{-0.8cm}
	\begin{multicols}{2}
	\justify
		Docker es una herramienta que nos facilita enpaquetar una aplicación junto con su \textit{runtime} dentro de una \textbf{imagen}, que luego se ejecutará dentro de un \textbf{container}.

		Gracias a la gran cantidad de \textbf{imágenes base} que hay disponibles en el \textbf{Docker Hub} y la simple sintaxis del \textbf{Dockerfile}, crear imágenes propias con nuestra aplicación es muy simple.
	\columnbreak
	\vspace*{\fill}
		\textbf{Dockerfile}
		\begin{minted}{docker}
FROM debian:9
RUN echo "echo 'Hola mundo!'"\
		> /hi.sh && \
		chmod +x /hi.sh
ENTRYPOINT hi.sh
		\end{minted}
	\vspace*{\fill}
	\end{multicols}
\end{frame}

\begin{frame}[fragile]
	\frametitle{¿Qué es un container?}
	\vspace{-0.8cm}
	\begin{multicols}{2}
	\justify
			\begin{spacing}{0.8}
		\small Un \textbf{container} es una funcionalidad del sistema operativo que nos permite ejecutar ciertas aplicaciones dentro de un \textbf{namespace} aparte.

		\vspace{0.25cm}

		\small Un \textbf{namespace} es una separación lógica de los recursos del sistema, como por ejemplo CPU, memoria RAM, puntos de montaje (disco), árbol de procesos, network, etc.

		\vspace{0.25cm}

		\small De esta forma un container nos provee seguridad y aislamiento de las aplicaciones que corren en él, similar a lo que nos permite una VM, sin el overhead de simular hardware y correr un sistema operativo entero, ya que todos los containers corriendo en un mismo host comparten su kernel.
			\end{spacing}
	\columnbreak
	\vspace*{\fill}
		\includegraphics[height=.6\textheight]{assets/containerz.png}
	\vspace*{\fill}
	\end{multicols}
\end{frame}

\begin{frame}[fragile]
	\frametitle{El concepto de "layer"}
	\vspace{-0.5cm}
	\justify
	\centering
	\includegraphics[width=.75\textwidth]{assets/layers3.png}
	\vspace{-0.5cm}
	\begin{multicols}{2}
	\justify
		\begin{spacing}{0.8}
		\small Las imágenes de Docker que podemos usar a diario están compuestas por una o más \textbf{layers}. Éstas imágenes contienen un sistema de archivos correspondiente al sistema operativo o la aplicación que queremos ejecutar. Esta imagen está identificada por un hash y opcionalmente uno o más \textbf{tags}, como por ejemplo \textit{redis:4.0.6} o \textit{debian:latest}.

		\columnbreak

		\small Cada layer contiene únicamente los cambios (ya sean archivos, comandos o metadatos) con respecto a la layer anterior, se identifican con un hash y corresponden cada una a un comando del Dockerfile. Todas las layers que componen una imagen son \textbf{read only}.

		\end{spacing}
	\end{multicols}
\end{frame}

\begin{frame}[fragile]
	\frametitle{El concepto de "layer"}
	\vspace{-0.8cm}
	\begin{multicols}{2}
	\justify
	\vspace*{\fill}
		\hspace{-0.5cm}
		\includegraphics[height=.5\textheight]{assets/layers2.jpg}
	\vspace*{\fill}
	\columnbreak
	\vspace*{\fill}
		\begin{spacing}{0.8}
		\small Cada layer contiene únicamente los cambios (ya sean archivos, comandos o metadatos) con respecto a la layer anterior, se identifican con un hash y corresponden cada una a un comando del Dockerfile. Todas las layers que componen una imagen son \textbf{read only}.

		\vspace{0.25cm}

		\small Al momento de crear un nuevo container a base de una imagen, se crea una pequeña layer \textbf{read-write}, que contiene los cambios creados por el container en ejecución (por ejemplo, por la aplicación).

		\vspace{0.25cm}

		\small Esta última layer se \textbf{pierde al eliminar el container}.
			\end{spacing}
	\vspace*{\fill}
	\end{multicols}
\end{frame}

%% Fin de la presentación!
\begin{frame}
	\vspace{-0.8cm}
	\begin{center}
		%% Foto de Perón
		\begin{figure}
		\includegraphics[width = 0.8\textwidth]{./assets/peron1.jpg}
		\end{figure}
		\Large{\textbf{Gracias! \\ Preguntas?}}
	\end{center}
\end{frame}

\end{document}
